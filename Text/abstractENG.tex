\noindent As smart cities are expanding, more and more problems concerning the energy consumption for Wireless Sensor Networks (WSNs) are arisen. In this work a WSN is defined as a two dimensional area which contains sensor nodes and a mobile sink, i.e. a movable base station. Since the area of such a network could be large and contain many nodes, it is of interest to compare two new energy methods within two different fields. \newline

\noindent This thesis work investigated a new method to cluster wireless sensor nodes together with two new control approaches for a mobile sink. In an WSN environment developed by the authors, a model based controller and a data driven controller for controlling a mobile sink was designed and deployed. By controlling the placement of the sink and the packet rate of all nodes the controllers were able to reduce the energy consumption of the entire system. \newline

\noindent The clustering method and two control approaches were compared on the metrics of energy consumed, data packets sent and the amount of data that can be sent with an energy unit. From tests conducted on the system the results show \dots \newline   


\noindent \textcolor{red}{TODO: Update with results and short discussion. Potentially write a short introduction to what a WSN is.}\newline

\noindent \textit{Keywords: Wireless sensor network, internet of things, nodes, mobile sink, base station, data packets, embedded systems, energy management, clustering, LEACH, cluster head, model based controller, MPC, optimization, data driven control, reinforcement learning, DQN, neural networks}